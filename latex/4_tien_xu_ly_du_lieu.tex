\section{Preprocessing Data (Tiền xử lý dữ liệu)}

\subsection{Giới thiệu chung về vai trò của tiền xử lý dữ liệu trong khai thác dữ liệu}

\hspace{0.5cm}Trong bất kỳ dự án khai thác dữ liệu nào, tiền xử lý dữ liệu đóng vai trò quan trọng vì dữ liệu thô (\textit{raw data}) thường không đầy đủ, không chính xác, hoặc không nhất quán. Quá trình tiền xử lý giúp cải thiện chất lượng dữ liệu trước khi áp dụng các mô hình học máy và khai thác dữ liệu, từ đó giúp tăng độ chính xác của các dự đoán hoặc phân tích.

\subsubsection{Mục đích của tiền xử lý dữ liệu}

\hspace{0.5cm}Mục đích chính của tiền xử lý dữ liệu là làm sạch và chuẩn hóa dữ liệu, giảm thiểu nhiễu và những giá trị không hợp lệ, từ đó nâng cao chất lượng của dữ liệu. Việc này cải thiện hiệu quả của các mô hình học máy và khai thác dữ liệu, giúp các thuật toán có thể học và dự đoán chính xác hơn.

\subsubsection{Quá trình tiền xử lý dữ liệu}

\hspace{0.5cm}Dữ liệu thô cần được xử lý để nâng cao chất lượng: Dữ liệu thô có thể chứa các giá trị thiếu (\textit{missing values}), dữ liệu nhiễu (\textit{noisy data}), hoặc dữ liệu không nhất quán (\textit{inconsistent data}), điều này có thể làm sai lệch kết quả phân tích. Do đó, tiền xử lý dữ liệu là bước cần thiết trước khi sử dụng dữ liệu cho các mô hình học máy.

\subsubsection{Các vấn đề thường gặp trong dữ liệu thô}

\begin{itemize}
    \item \textbf{Thiếu giá trị:} Nhiều thuộc tính hoặc cột trong dữ liệu có thể thiếu thông tin.
	
    \item \textbf{Nhiễu:} Các giá trị không hợp lý hoặc ngoại lệ (\textit{outliers}) có thể xuất hiện trong dữ liệu.
	
    \item \textbf{Không nhất quán:} Dữ liệu có thể được ghi nhận theo các cách khác nhau (ví dụ: định dạng ngày tháng, mã hóa giá trị).
\end{itemize}

\subsection{Ý nghĩa các trường dữ liệu}

\begin{description}
    \item[Dataset: Climate Change Impact on Agriculture 2024]

    \begin{itemize} 
        \item \textbf{Mô tả:} Dữ liệu này ghi nhận thông tin về tác động của biến đổi khí hậu đến nông nghiệp ở các quốc gia qua các năm.
        \item \textbf{Thuộc tính:}
        \begin{itemize}
            \item \textbf{Year:} Năm thu thập dữ liệu (1990-2024).
            \begin{itemize}
                \item \textbf{Vai trò:} Cung cấp thông tin theo thời gian, giúp phân tích xu hướng tác động của biến đổi khí hậu qua các năm.
            \end{itemize}
            \item \textbf{Country:} Quốc gia nghiên cứu.
            \begin{itemize}
                \item \textbf{Vai trò:} Phân loại dữ liệu theo quốc gia, giúp so sánh tác động của biến đổi khí hậu giữa các khu vực khác nhau.
            \end{itemize}
            \item \textbf{Region:} Khu vực trong quốc gia.
            \begin{itemize}
                \item \textbf{Vai trò:} Phân tích chi tiết hơn về tác động của biến đổi khí hậu ở cấp độ khu vực.
            \end{itemize}
            \item \textbf{Crop\_Type:} Loại cây trồng.
            \begin{itemize}
                \item \textbf{Vai trò:} Phân loại dữ liệu theo loại cây trồng, giúp đánh giá tác động của biến đổi khí hậu đến từng loại cây.
            \end{itemize}
            \item \textbf{Average\_Temperature\_C:} Nhiệt độ trung bình (đơn vị: \textdegree C).
            \begin{itemize}
                \item \textbf{Vai trò:} Đánh giá tác động của nhiệt độ đến nông nghiệp.
            \end{itemize}
            \item \textbf{Total\_Precipitation\_mm:} Tổng lượng mưa (đơn vị: mm).
            \begin{itemize}
                \item \textbf{Vai trò:} Phân tích ảnh hưởng của lượng mưa đến nông nghiệp.
            \end{itemize}
            \item \textbf{CO2\_Emissions\_MT:} Lượng khí thải CO2 (đơn vị: triệu tấn).
            \begin{itemize}
                \item \textbf{Vai trò:} Đánh giá tác động của khí thải nhà kính.
            \end{itemize}
            \item \textbf{Crop\_Yield\_MT\_per\_HA:} Năng suất cây trồng (đơn vị: tấn/ha).
            \begin{itemize}
                \item \textbf{Vai trò:} Đo lường hiệu quả sản xuất nông nghiệp.
            \end{itemize}
            \item \textbf{Extreme\_Weather\_Events:} Số lượng sự kiện thời tiết cực đoan.
            \begin{itemize}
                \item \textbf{Vai trò:} Đánh giá tác động của các hiện tượng thời tiết cực đoan.
            \end{itemize}
            \item \textbf{Irrigation\_Access\_\%:} Tỷ lệ tiếp cận tưới tiêu (đơn vị: \%).
            \begin{itemize}
                \item \textbf{Vai trò:} Phân tích khả năng thích ứng với biến đổi khí hậu.
            \end{itemize}
            \item \textbf{Pesticide\_Use\_KG\_per\_HA:} Lượng thuốc trừ sâu sử dụng (đơn vị: kg/ha).
            \begin{itemize}
                \item \textbf{Vai trò:} Đánh giá tác động của việc sử dụng thuốc trừ sâu.
            \end{itemize}
            \item \textbf{Fertilizer\_Use\_KG\_per\_HA:} Lượng phân bón sử dụng (đơn vị: kg/ha).
            \begin{itemize}
                \item \textbf{Vai trò:} Phân tích tác động của việc sử dụng phân bón.
            \end{itemize}
            \item \textbf{Soil\_Health\_Index:} Chỉ số sức khỏe đất (thang điểm 0-100).
            \begin{itemize}
                \item \textbf{Vai trò:} Đánh giá chất lượng đất canh tác.
            \end{itemize}
            \item \textbf{Adaptation\_Strategies:} Chiến lược thích ứng.
            \begin{itemize}
                \item \textbf{Vai trò:} Phân tích các biện pháp thích ứng với biến đổi khí hậu.
            \end{itemize}
            \item \textbf{Economic\_Impact\_Million\_USD:} Tác động kinh tế (đơn vị: triệu USD).
            \begin{itemize}
                \item \textbf{Vai trò:} Đánh giá tác động kinh tế của biến đổi khí hậu.
            \end{itemize}
        \end{itemize}
    \end{itemize}
\end{description}

\subsection{Quá trình tiền xử lý dữ liệu}

\subsubsection{Đọc và kiểm tra dữ liệu}

\hspace{0.5cm}Quá trình tiền xử lý dữ liệu bắt đầu bằng việc đọc và kiểm tra dữ liệu từ file CSV. Kết quả kiểm tra ban đầu cho thấy:

\begin{itemize}
    \item \textbf{Kích thước dữ liệu:} 10,000 dòng và 15 cột
    \item \textbf{Giá trị thiếu:} Không có giá trị thiếu trong bất kỳ cột nào
    \item \textbf{Dữ liệu trùng lặp:} Không phát hiện dữ liệu trùng lặp
\end{itemize}

\subsubsection{Xử lý dữ liệu thiếu và trùng lặp}

\hspace{0.5cm}Mặc dù không có giá trị thiếu trong dữ liệu, hệ thống vẫn được cấu hình để xử lý các trường hợp có thể xảy ra:

\begin{itemize}
    \item \textbf{Xử lý dữ liệu thiếu:}
    \begin{itemize}
        \item Cột số: Thay thế bằng giá trị trung bình (mean)
        \item Cột phân loại: Thay thế bằng giá trị xuất hiện nhiều nhất (mode)
        \item Cột quan trọng (Year, Country): Loại bỏ hàng có giá trị thiếu
    \end{itemize}
    
    \item \textbf{Xử lý dữ liệu trùng lặp:}
    \begin{itemize}
        \item Tự động phát hiện và loại bỏ các hàng trùng lặp
        \item Ghi log số lượng hàng bị loại bỏ
    \end{itemize}
\end{itemize}

\subsubsection{Kỹ thuật biến đổi đặc trưng}

\hspace{0.5cm}Hệ thống thực hiện các biến đổi đặc trưng để chuẩn bị dữ liệu cho việc phân tích:

\begin{itemize}
    \item \textbf{Xử lý cột số:}
    \begin{itemize}
        \item Xử lý ngoại lệ (outliers) với các chiến lược khác nhau:
        \begin{itemize}
            \item Clipping cho các cột có giới hạn rõ ràng:
            \begin{itemize}
                \item Year: 1990-2024
                \item Irrigation\_Access\_\%: 0-100\%
                \item Soil\_Health\_Index: 0-100
                \item Extreme\_Weather\_Events: 0-10
            \end{itemize}
            \item IQR method cho các cột khác
        \end{itemize}
        \item Chuẩn hóa dữ liệu sử dụng MinMaxScaler cho 10 cột số
    \end{itemize}
    
    \item \textbf{Xử lý cột phân loại:}
    \begin{itemize}
        \item Tạo biến giả (dummy variables) cho 4 cột:
        \begin{itemize}
            \item Country
            \item Region
            \item Crop\_Type
            \item Adaptation\_Strategies
        \end{itemize}
        \item Giới hạn tối đa 10 hạng mục cho mỗi đặc trưng
        \item Tạo category "Other" cho các giá trị ít xuất hiện
    \end{itemize}
\end{itemize}

\subsubsection{Kết quả xử lý}

\hspace{0.5cm}Sau quá trình tiền xử lý, dữ liệu có những thay đổi sau:

\begin{itemize}
    \item \textbf{Kích thước dữ liệu:} Tăng từ 15 cột lên 46 cột
    \begin{itemize}
        \item 11 cột số gốc đã được chuẩn hóa
        \item 35 cột boolean từ việc tạo biến giả
    \end{itemize}
    \item \textbf{Chất lượng dữ liệu:}
    \begin{itemize}
        \item Không có giá trị thiếu
        \item Không có dữ liệu trùng lặp
        \item Các giá trị ngoại lệ đã được xử lý
        \item Dữ liệu số đã được chuẩn hóa về khoảng [0,1]
    \end{itemize}
\end{itemize}

\subsection*{Tóm Tắt}
Mỗi dataset có các thuộc tính đặc trưng riêng, và trong quá trình tiền xử lý, nhóm sẽ làm sạch, chuẩn hóa và tích hợp các thuộc tính này để đảm bảo dữ liệu có chất lượng tốt nhất cho việc phân tích và xây dựng mô hình học máy. Việc hiểu rõ vai trò của từng thuộc tính trong dữ liệu sẽ giúp xác định cách thức xử lý và cải thiện chất lượng dữ liệu hiệu quả hơn.