\section{Giới thiệu dự án}

Trong bối cảnh nền kinh tế Việt Nam đang ngày càng hội nhập sâu rộng với thế giới, nông nghiệp vẫn giữ vai trò then chốt trong việc đảm bảo an ninh lương thực và phát triển bền vững. Tuy nhiên, giá nông sản luôn biến động khó lường do chịu ảnh hưởng bởi nhiều yếu tố như thời tiết, dịch bệnh, chi phí vận chuyển, chính sách thị trường và nhu cầu tiêu dùng. Việc phân tích và dự đoán giá nông sản một cách chính xác không chỉ giúp nông dân và doanh nghiệp nông nghiệp đưa ra quyết định sản xuất, kinh doanh hiệu quả mà còn hỗ trợ cơ quan quản lý nhà nước trong công tác hoạch định chính sách.

Do đó, đề tài \textbf{"Phân tích các yếu tố ảnh hưởng tới sản lượng nông sản và xây dựng mô hình dự đoán"} mang tính cần thiết và cấp thiết trong thực tiễn. Bằng việc áp dụng các phương pháp phân tích và học máy hiện đại, đề tài hướng tới mục tiêu không chỉ tìm hiểu mối quan hệ giữa các yếu tố ảnh hưởng mà còn xây dựng mô hình dự đoán có độ chính xác cao, hỗ trợ người dùng trong việc đưa ra quyết định kịp thời.

Để đạt được mục tiêu đó, nhóm đề tài sử dụng các phương pháp phân lớp dữ liệu phổ biến trong lĩnh vực học máy, bao gồm:

\begin{itemize}
    \item \textbf{Hồi quy tuyến tính (Linear Regression)}: Mô hình dự đoán tuyến tính đơn giản nhưng hiệu quả với dữ liệu có mối quan hệ tuyến tính.
    \item \textbf{Cây quyết định (Decision Tree)}: Phương pháp phân loại dữ liệu theo dạng cây nhị phân, dễ diễn giải và trực quan.
    \item \textbf{Xác suất Bayes (Naive Bayes)}: Phương pháp dựa trên định lý Bayes, phù hợp với dữ liệu có tính độc lập tương đối giữa các đặc trưng.
    \item \textbf{Mạng nơ-ron nhân tạo (Artificial Neural Network - ANN)}: Mô hình mô phỏng hoạt động của nơ-ron sinh học, có khả năng học sâu và phi tuyến.
    \item \textbf{K-nearest neighbors (KNN)}: Phương pháp phân loại dựa trên khoảng cách với các điểm lân cận gần nhất trong không gian đặc trưng.
\end{itemize}

Với cách tiếp cận đa phương pháp như trên, đề tài kỳ vọng sẽ đưa ra được đánh giá toàn diện về các yếu tố ảnh hưởng và lựa chọn được mô hình dự đoán giá nông sản phù hợp nhất với dữ liệu thực tế.
